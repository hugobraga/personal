%%%%%%%%%%%%%%%%%%%%%%%%%%%%%%%%%%%%%%%%%%%%%%%%%%%%%%%%%%%%%%%%%%%%%%%%
%%%%%%%%%%%%%%%%%%%%%% Simple LaTeX CV Template %%%%%%%%%%%%%%%%%%%%%%%%
%%%%%%%%%%%%%%%%%%%%%%%%%%%%%%%%%%%%%%%%%%%%%%%%%%%%%%%%%%%%%%%%%%%%%%%%

%%%%%%%%%%%%%%%%%%%%%%%%%%%%%%%%%%%%%%%%%%%%%%%%%%%%%%%%%%%%%%%%%%%%%%%%
%% NOTE: If you find that it says                                     %%
%%                                                                    %%
%%                           1 of ??                                  %%
%%                                                                    %%
%% at the bottom of your first page, this means that the AUX file     %%
%% was not available when you ran LaTeX on this source. Simply RERUN  %%
%% LaTeX to get the ``??'' replaced with the number of the last page  %%
%% of the document. The AUX file will be generated on the first run   %%
%% of LaTeX and used on the second run to fill in all of the          %%
%% references.                                                        %%
%%%%%%%%%%%%%%%%%%%%%%%%%%%%%%%%%%%%%%%%%%%%%%%%%%%%%%%%%%%%%%%%%%%%%%%%

%%%%%%%%%%%%%%%%%%%%%%%%%%%% Document Setup %%%%%%%%%%%%%%%%%%%%%%%%%%%%

% Don't like 10pt? Try 11pt or 12pt
\documentclass[10pt]{article}
\RequirePackage[T1]{fontenc}

% LaTeX will typeset using Computer Modern Roman, which a lot of
% non-mathematicians and non-engineers won't like. Also, a few PDF
% viewers may not render CMR very well. Instead, Times New Roman can
% be used. That's what this package does.
\usepackage{times}

% The automated optical recognition software used to digitize resume
% information works best with fonts that do not have serifs. This
% command uses a sans serif font throughout. Uncomment both lines (or at
% least the second) to restore a Roman font (i.e., a font with serifs).
% (NOTE: This requires the times package above)
%\renewcommand{\familydefault}{\sfdefault}

% This is a helpful package that puts math inside length specifications
\usepackage{calc}

% This package helps LaTeX auto-hyphenate hyphenated words if you use
% special hyphens. For example, bio\-/mimicry will properly hyphenate
% ``mimicry'' if necessary.
\usepackage[shortcuts]{extdash}

% Layout: Puts the section titles on left side of page
\reversemarginpar

%
%         PAPER SIZE, PAGE NUMBER, AND DOCUMENT LAYOUT NOTES:
%
% The next \usepackage line changes the layout for CV style section
% headings as marginal notes. It also sets up the paper size as either
% letter or A4. By default, letter was used. If A4 paper is desired,
% comment out the letterpaper lines and uncomment the a4paper lines.
%
% As you can see, the margin widths and section title widths can be
% easily adjusted.
%
% ALSO: Notice that the includefoot option can be commented OUT in order
% to put the PAGE NUMBER *IN* the bottom margin. This will make the
% effective text area larger.
%
% IF YOU WISH TO REMOVE THE ``of LASTPAGE'' next to each page number,
% see the note about the +LP and -LP lines below. Comment out the +LP
% and uncomment the -LP.
%
% IF YOU WISH TO REMOVE PAGE NUMBERS, be sure that the includefoot line
% is uncommented and ALSO uncomment the \pagestyle{empty} a few lines
% below.
%

%% Use these lines for letter-sized paper
\usepackage[paper=letterpaper,
            %includefoot, % Uncomment to put page number above margin
            marginparwidth=1.2in,     % Length of section titles
            marginparsep=.05in,       % Space between titles and text
            margin=1in,               % 1 inch margins
            includemp]{geometry}

%% Use these lines for A4-sized paper
%\usepackage[paper=a4paper,
%            %includefoot, % Uncomment to put page number above margin
%            marginparwidth=30.5mm,    % Length of section titles
%            marginparsep=1.5mm,       % Space between titles and text
%            margin=25mm,              % 25mm margins
%            includemp]{geometry}

%% More layout: Get rid of indenting throughout entire document
\setlength{\parindent}{0in}

% Provides special list environments and macros to create new ones
\usepackage[shortlabels]{enumitem}

% Simpler bibsections for CV sections
% (thanks to natbib for inspiration)
%
% * For lists of references with hanging indents and no numbers:
%
%   \begin{bibsection}
%       \item ...
%   \end{bibsection}
%
% * For numbered lists of references (with hanging indents):
%
%   \begin{bibenum}
%       \item ...
%   \end{bibenum}
%
%   Note that bibenum numbers continuously throughout. To reset the
%   counter, use
%
%   \restartlist{bibenum}
%
%   at the place where you want the numbering to reset.

%extraido do resumo de Ken Mankoff
% http://tex.stackexchange.com/questions/9768/multibib-reverse-label-or-sort-order
%the sorting parameter allows me to not need to sort the references in bib file
%% \usepackage[backend=biber,style=numeric, defernumbers=true,
%% firstinits=true, maxnames=9, sorting=ydnt]{biblatex}

%----hugo-------
%% \usepackage[backend=biber,style=numeric, defernumbers=true,
%%   giveninits=true, maxnames=9, sorting=ydnt]{biblatex}
\usepackage{csquotes}
\usepackage[english]{babel}
%\printbibliography[heading=none]
%% \usepackage{etoolbox}
%% \patchcmd{\thebibliography}{\section*{\refname}}{}{}{}
%% \usepackage{polyglossia}
%% \setmainlanguage{english}
%% \setotherlanguage[variant=brazilian]{portuguese}
\usepackage{refcount}%------12/02/20----------
\usepackage[
    backend=biber,
    style=numeric,
    giveninits=true,
    maxcitenames=99,
    maxbibnames=99,
    sorting=ydnt,
    %sorting=adnyvt,
    sortlocale=en-US,
    defernumbers=true,
    refsegment=section,
    isbn=false,
]{biblatex}
\addbibresource{HugoBraga-CV-eng.bib}
%---------------

% http://tex.stackexchange.com/questions/9768/multibib-reverse-label-or-sort-order
%% \AtDataInput{%
%%   \csnumgdef{entrycount:\strfield{prefixnumber}}{%
%%     \csuse{entrycount:\strfield{prefixnumber}}+1}}
%% \DeclareFieldFormat{labelnumber}{\mkbibdesc{#1}}
%% \newrobustcmd*{\mkbibdesc}[1]{%
%%   \number\numexpr\csuse{entrycount:\strfield{prefixnumber}}+1-#1\relax}

%\bibliography{hugo.bib}

% remove ``in''
\renewbibmacro{in:}{}

% make the citations ([42]) smaller. Use \tiny,\scriptsize,\footnotesize, or \small
\let\oldcite=\cite
\renewcommand{\cite}[1]{\footnotesize{\textbf{\oldcite{#1}}}\normalsize{}}

%%% uncomment to hide citations
%\renewcommand{\cite}[2][]{\unskip}

%\nocite{*} %29/03/20
%fim da extraido do resumo de Ken Mankoff

%------hugo-----------29/02/20
%% \makeatletter
%% \patchcmd{\blx@printbibliography}
%%   {\blx@bibliography\blx@tempa}
%%   {\setcounter{bibitemtotal}{0}%
%%    \begingroup
%%    \def\do##1{\stepcounter{bibitemtotal}}%
%%    \dolistloop{\blx@tempa}%
%%    \endgroup
%%    \blx@bibliography\blx@tempa}{}{}
%% \makeatother

%% \newcounter{bibitemtotal}
%% \newrobustcmd*{\mkbibdesc}[1]{%
%%   \number\numexpr\value{bibitemtotal}+1-#1\relax}
%% \DeclareFieldFormat{labelnumber}{\mkbibdesc{#1}}
%% \DeclareFieldFormat{labelnumberwidth}{\mkbibbrackets{#1}}

%% \defbibenvironment{bibliography}
%%   {\list
%%      {\printtext[labelnumberwidth]{\printfield{labelprefix}\printfield{labelnumber}}}
%%      {}%
%%       \renewcommand*{\makelabel}[1]{\hss##1}}
%%   {\endlist}
%%   {\item}
  %------fim hugo-------
  
%% \AtDataInput{%
%%   \csnumgdef{entrycount:\strfield{prefixnumber}}{%
%%     \csuse{entrycount:\strfield{prefixnumber}}+1}}
%% \DeclareFieldFormat{labelnumber}{\mkbibdesc{#1}}
%% \newrobustcmd*{\mkbibdesc}[1]{%
%%   \number\numexpr\csuse{entrycount:\strfield{prefixnumber}}+1-#1\relax}

%% \makeatletter
%% %\newlength{\bibhang} %ja esta no biblatex.sty
%% \setlength{\bibhang}{1em}
%% \newlength{\bibsep}
%%  {\@listi \global\bibsep\itemsep \global\advance\bibsep by\parsep}
%% \newlist{bibsection}{itemize}{3}
%% \setlist[bibsection]{label=,leftmargin=\bibhang,%
%%         itemindent=-\bibhang,
%%         itemsep=\bibsep,parsep=\z@,partopsep=0pt,
%%         topsep=0pt}
%% \newlist{bibenum}{enumerate}{3}
%% \setlist[bibenum]{label=[\arabic*],resume,leftmargin={\bibhang+\widthof{[999]}},%
%%         itemindent=-\bibhang,
%%         itemsep=\bibsep,parsep=\z@,partopsep=0pt,
%%         topsep=0pt}
%% \let\oldendbibenum\endbibenum
%% \def\endbibenum{\oldendbibenum\vspace{-.6\baselineskip}}
%% \let\oldendbibsection\endbibsection
%% \def\endbibsection{\oldendbibsection\vspace{-.6\baselineskip}}
%% \makeatother

%%apagando----------12/02/20--------
%----05/04/20----------
%% \AtDataInput[publication]{
%%  \xifinlistcs{\thefield{entrykey}}{entrylist:\therefsection}{}{
%%   \listcsxadd{entrylist:\therefsection}{\thefield{entrykey}}
%%   \csnumgdef{publicationentrycount:\therefsection}{
%%       \csuse{publicationentrycount:\therefsection}+1}}}

%% \DeclareFieldFormat[publication]{labelnumber}{\mkbibpublicationdesc{#1}}

%% \newrobustcmd*{\mkbibpublicationdesc}[1]{
%%   \number\numexpr\csuse{publicationentrycount:\therefsection}+1-#1\relax}
%----05/04/20----------

%-----12/04/20
%https://tex.stackexchange.com/questions/327408/biblatex-descending-order-of-references-with-labels-not-working
%% \makeatletter
%% \patchcmd{\blx@printbibliography}
%%   {\blx@bibliography\blx@tempa}
%%   {\setcounter{otherentrycount}{0}%
%%    \begingroup
%%    \def\do##1{\stepcounter{otherentrycount}}%
%%    \dolistloop{\blx@tempa}%
%%    \endgroup
%%    \blx@bibliography\blx@tempa}{}{}
%% \makeatother

%% \newcounter{otherentrycount}
%% \newrobustcmd*{\mkbibdesc}[1]{%
%%   \number\numexpr\value{otherentrycount}+1-#1\relax}
%% \DeclareFieldFormat{labelnumber}{\mkbibdesc{#1}}
%% \DeclareFieldFormat{labelnumberwidth}{\mkbibbrackets{#1}}

%% \defbibenvironment{bibliography}
%%   {\list
%%      {\printtext[labelnumberwidth]{\printfield{labelprefix}\printfield{labelnumber}}}
%%      {}%
%%       \renewcommand*{\makelabel}[1]{\hss##1}}
%%   {\endlist}
%%   {\item}
%-----  12/04/20

%-----12/04/20----------------
%% \DeclareBibliographyCategory{primary}
%% \DeclareBibliographyCategory{secondary}
%% \defbibheading{primary}{\subsection*{Primary sources}}
%% \defbibheading{secondary}{\subsection*{Secondary sources}}

%% \newcounter{publicationcount}
%% \newcounter{secondary}
%% \AtDataInput{%
%%   \ifboolexpr{ test {\ifkeyword{publication}} } {
%%     \addtocategory{primary}{\thefield{entrykey}}%
%%     \stepcounter{publicationcount}
%%   }
%%   {}%else
%%   \ifboolexpr{ test {\ifkeyword{secondary}} and not test {\iffieldundef{year}}
%%     and test {\ifnumless{\thefield{year}+0}{1980}} } {
%%     \addtocategory{secondary}{\thefield{entrykey}}%
%%     \stepcounter{secondary}
%%   }
%%   {}%else
%% }

\AtDataInput{%
  \ifkeyword{publication} {
%  \ifboolexpr{ test {\ifkeyword{publication}} } {
    \csnumgdef{publicationentrycount:\therefsection:\thefield{entrykey}} {%
      \csuse{publicationentrycount:\therefsection:\thefield{entrykey}}+1
    }
  }
  {}%else

  \ifkeyword{grant} {
%  \ifboolexpr{ test {\ifkeyword{grant}} } {
    \csnumgdef{grantentrycount:\therefsection:\thefield{entrykey}} {%
      \csuse{grantentrycount:\therefsection:\thefield{entrykey}}+1
    }
  }
  {}%else

  \ifkeyword{others} {
%  \ifboolexpr{ test {\ifkeyword{others}} } {
    \csnumgdef{otherentrycount:\therefsection:\thefield{entrykey}} {%
      \csuse{otherentrycount:\therefsection:\thefield{entrykey}}+1
    }
  }
  {}%else
  
}


%% \AtDataInput[publication]{%
%%   \csnumgdef{publicationentrycount:\therefsection:\thefield{entrykey}} {%
%%     \csuse{publicationentrycount:\therefsection:\thefield{entrykey}}+1
%%   }
%% }

\DeclareFieldFormat[publication]{labelnumber}{\mkbibpublicationdesc{#1}}    
\newrobustcmd*{\mkbibpublicationdesc}[1]{%
  \number\numexpr\csuse{publicationentrycount:\therefsection:\thefield{entrykey}}+1-#1\relax}


%% \AtDataInput[other]{%
%%   \csnumgdef{otherentrycount:\therefsection:\thefield{entrykey}}{%
%%     \csuse{otherentrycount:\therefsection:\thefield{entrykey}}+1}}

\DeclareFieldFormat[others]{labelnumber}{\mkbibotherdesc{#1}}    
\newrobustcmd*{\mkbibotherdesc}[1]{%
  \number\numexpr\csuse{otherentrycount:\therefsection:\thefield{entrykey}}+1-#1\relax}


%% \AtDataInput[grant]{%
%%   \csnumgdef{grantentrycount:\therefsection:\thefield{entrykey}}{%
%%     \csuse{grantentrycount:\therefsection:\thefield{entrykey}}+1}}

\DeclareFieldFormat[grant]{labelnumber}{\mkbibgrantdesc{#1}}    
\newrobustcmd*{\mkbibgrantdesc}[1]{%
  \number\numexpr\csuse{grantentrycount:\therefsection:\thefield{entrykey}}+1-#1\relax}


%% \AtDataInput{%
%%   \csnumgdef{entrycount:\therefsection:\thefield{entrykey}}{%
%%     \csuse{entrycount:\therefsection:\thefield{entrykey}}+1}}

%% \DeclareFieldFormat{labelnumber}{\mkbibdesc{#1}}    
%% \newrobustcmd*{\mkbibdesc}[1]{%
%%   \number\numexpr\csuse{entrycount:\therefsection:\thefield{entrykey}}+1-#1\relax}
%-----12/04/20----------------

%-----05/04/20----------------
%% \AtDataInput[other]{
%%  \xifinlistcs{\thefield{entrykey}}{entrylist:\therefsection}{}{
%%   \listcsxadd{entrylist:\therefsection}{\thefield{entrykey}}
%%   \csnumgdef{otherentrycount:\therefsection}{
%%     \csuse{otherentrycount:\therefsection}+1}}}

%% \DeclareFieldFormat[other]{labelnumber}{\mkbibotherdesc{#1}}
%% \DeclareFieldFormat[other]{labelnumberwidth}{\mkbibbrackets{#1}}

%% \newrobustcmd*{\mkbibotherdesc}[1]{
%%   \number\numexpr\value{otherentrycount:\therefsection}+1-#1\relax}

%% \AtDataInput[grant]{
%%  \xifinlistcs{\thefield{entrykey}}{entrylist:\therefsection}{}{
%%   \listcsxadd{entrylist:\therefsection}{\thefield{entrykey}}
%%   \csnumgdef{grantentrycount:\therefsection}{
%%       \csuse{grantentrycount:\therefsection}+1}}}

%% \DeclareFieldFormat[grant]{labelnumber}{\mkbibgrantdesc{#1}}

%% \newrobustcmd*{\mkbibgrantdesc}[1]{
%%    \number\numexpr\csuse{grantentrycount:\therefsection}+1-#1\relax}
%----05/04/20----------
%%apagando----------12/02/20--------

%--------------------

%% Reference the last page in the page number
%
% NOTE: comment the +LP line and uncomment the -LP line to have page
%       numbers without the ``of ##'' last page reference)
%
% NOTE: uncomment the \pagestyle{empty} line to get rid of all page
%       numbers (make sure includefoot is commented out above)
%
\usepackage{fancyhdr,lastpage}
\pagestyle{fancy}
%\pagestyle{empty}      % Uncomment this to get rid of page numbers
\fancyhf{}\renewcommand{\headrulewidth}{0pt}
\fancyfootoffset{\marginparsep+\marginparwidth}
\newlength{\footpageshift}
\setlength{\footpageshift}
          {0.5\textwidth+0.5\marginparsep+0.5\marginparwidth-2in}
\lfoot{\hspace{\footpageshift}%
       \parbox{4in}{\, \hfill %
                    \arabic{page} of \protect\pageref*{LastPage} % +LP
%                    \arabic{page}                               % -LP
                    \hfill \,}}

% Finally, give us PDF bookmarks
\usepackage{color,hyperref}
\definecolor{darkblue}{rgb}{0.0,0.0,0.3}
\hypersetup{colorlinks,breaklinks,
            linkcolor=darkblue,urlcolor=darkblue,
            anchorcolor=darkblue,citecolor=darkblue}

%%%%%%%%%%%%%%%%%%%%%%%% End Document Setup %%%%%%%%%%%%%%%%%%%%%%%%%%%%


%%%%%%%%%%%%%%%%%%%%%%%%%%% Helper Commands %%%%%%%%%%%%%%%%%%%%%%%%%%%%

%%% HEADING AT TOP OF CURRICULUM VITAE

% The title (name) with a horizontal rule under it
% (optional argument typesets an object right-justified across from name
%  as well)
%
% Usage: \makeheading{name}
%        OR
%        \makeheading[right_object]{name}
%
% Place at top of document. It should be the first thing.
% If ``right_object'' is provided in the square-braced optional
% argument, it will be right justified on the same line as ``name'' at
% the top of the CV. For example:
%
%       \makeheading[\emph{Curriculum vitae}]{Your Name}
%
% will put an emphasized ``Curriculum vitae'' at the top of the document
% as a title. Likewise, a picture could be included:
%
%   \makeheading[{\includegraphics[height=1.5in]{my_picture}}]{Your Name}
%
% the picture will be flush right across from the name. For this example
% to work, make sure the extra set of curly braces is included. Also
% makes ure that \usepackage{graphicx} is somewhere in the preamble.
\newcommand{\makeheading}[2][]%
        {\hspace*{-\marginparsep minus \marginparwidth}%
         \begin{minipage}[t]{\textwidth+\marginparwidth+\marginparsep}%
             {\large \bfseries #2 \hfill #1}\\[-0.15\baselineskip]%
                 \rule{\columnwidth}{1pt}%
         \end{minipage}}

%%% SECTION HEADINGS

% The section headings. Flush left in small caps down pseudo-margin.
%
% Usage: \section{section name}
\renewcommand{\section}[1]{\pagebreak[3]%
    \vspace{1.3\baselineskip}%
    \phantomsection\addcontentsline{toc}{section}{#1}%
    \noindent\llap{\scshape\smash{\parbox[t]{\marginparwidth}{\hyphenpenalty=10000\raggedright #1}}}%
    \vspace{-\baselineskip}\par}

%%% LISTS

% This macro alters a list by removing some of the space that follows the list
% (is used by lists below)
\newcommand*\fixendlist[1]{%
    \expandafter\let\csname preFixEndListend#1\expandafter\endcsname\csname end#1\endcsname
    \expandafter\def\csname end#1\endcsname{\csname preFixEndListend#1\endcsname\vspace{-0.6\baselineskip}}}

% These macros help ensure that items in outer-type lists do not get
% separated from the next line by a page break
% (they are used by lists below)
\let\originalItem\item
\newcommand*\fixouterlist[1]{%
    \expandafter\let\csname preFixOuterList#1\expandafter\endcsname\csname #1\endcsname
    \expandafter\def\csname #1\endcsname{\let\oldItem\item\def\item{\pagebreak[2]\oldItem}\csname preFixOuterList#1\endcsname}
    \expandafter\let\csname preFixOuterListend#1\expandafter\endcsname\csname end#1\endcsname
    \expandafter\def\csname end#1\endcsname{\let\item\oldItem\csname preFixOuterListend#1\endcsname}}
\newcommand*\fixinnerlist[1]{%
    \expandafter\let\csname preFixInnerList#1\expandafter\endcsname\csname #1\endcsname
    \expandafter\def\csname #1\endcsname{\let\oldItem\item\let\item\originalItem\csname preFixInnerList#1\endcsname}
    \expandafter\let\csname preFixInnerListend#1\expandafter\endcsname\csname end#1\endcsname
    \expandafter\def\csname end#1\endcsname{\csname preFixInnerListend#1\endcsname\let\item\oldItem}}

% An itemize-style list with lots of space between items
%
% Usage:
%   \begin{outerlist}
%       \item ...    % (or \item[] for no bullet)
%   \end{outerlist}
\newlist{outerlist}{itemize}{3}
    \setlist[outerlist]{label=\enskip\textbullet,leftmargin=*}
    \fixendlist{outerlist}
    \fixouterlist{outerlist}

% An environment IDENTICAL to outerlist that has better pre-list spacing
% when used as the first thing in a \section
%
% Usage:
%   \begin{lonelist}
%       \item ...    % (or \item[] for no bullet)
%   \end{lonelist}
\newlist{lonelist}{itemize}{3}
    \setlist[lonelist]{label=\enskip\textbullet,leftmargin=*,partopsep=0pt,topsep=0pt}
    \fixendlist{lonelist}
    \fixouterlist{lonelist}

% An itemize-style list with little space between items
%
% Usage:
%   \begin{innerlist}
%       \item ...    % (or \item[] for no bullet)
%   \end{innerlist}
\newlist{innerlist}{itemize}{3}
    \setlist[innerlist]{label=\enskip\textbullet,leftmargin=*,parsep=0pt,itemsep=0pt,topsep=0pt,partopsep=0pt}
    \fixinnerlist{innerlist}

% An environment IDENTICAL to innerlist that has better pre-list spacing
% when used as the first thing in a \section
%
% Usage:
%   \begin{loneinnerlist}
%       \item ...    % (or \item[] for no bullet)
%   \end{loneinnerlist}
\newlist{loneinnerlist}{itemize}{3}
    \setlist[loneinnerlist]{label=\enskip\textbullet,leftmargin=*,parsep=0pt,itemsep=0pt,topsep=0pt,partopsep=0pt}
    \fixendlist{loneinnerlist}
    \fixinnerlist{loneinnerlist}

%%% EXTRA SPACE

% To add some paragraph space between lines.
% This also tells LaTeX to preferably break a page on one of these gaps
% if there is a needed pagebreak nearby.
\newcommand{\blankline}{\quad\pagebreak[3]}
\newcommand{\halfblankline}{\quad\vspace{-0.5\baselineskip}\pagebreak[3]}

%%% FORMATTING MACROS

% Provides a linked \doi{#1} that links doi:#1 to http://dx.doi.org/#1
\usepackage{doi}
% To change the text before the DOI, adjust this command
%\renewcommand\doitext{doi:}

% Provides a linked \url{#1} that doesn't require escape characters
\usepackage{url}

% You can adjust the style \url{} uses here:
% (options are: same, rm, sf, tt; defaults to tt)
\urlstyle{same}

% For \email{ADDRESS}, links ADDRESS to the url mailto:ADDRESS
% (uncomment to typeset the e\-/mail address in typewriter font;
%  otherwise, will be typeset in the \urlstyle above)
%\DeclareUrlCommand\emaillink{\urlstyle{tt}}
\providecommand*\emaillink[1]{\nolinkurl{#1}}
\providecommand*\email[1]{\href{mailto:#1}{\emaillink{#1}}}

\providecommand\BibTeX{{B\kern-.05em{\sc i\kern-.025em b}\kern-.08em \TeX}}
\providecommand\Matlab{\textsc{Matlab}}

% Custom hyphenation rules for words that LaTeX has trouble with
\hyphenation{bio-mim-ic-ry bio-in-spi-ra-tion re-us-a-ble pro-vid-er Media-Wiki}

%%%%%%%%%%%%%%%%%%%%%%%% End Helper Commands %%%%%%%%%%%%%%%%%%%%%%%%%%%

%%%%%%%%%%%%%%%%%%%%%%%%% Begin CV Document %%%%%%%%%%%%%%%%%%%%%%%%%%%%

\begin{document}
\makeheading{Hugo V.~V.~Braga}

\section{Contact Information}

% NOTE: Mind where the & separators and \\ breaks are in the following
%       table. Table is one row made up of three parboxes. The left
%       parbox has address info, the middle parbox has a vertical bar,
%       and the right parbox has phone and electronic contact
%       information.
%
% MACROS: \rcollength is the width of the right column of the table
%             (adjust it to your liking; default is 1.85in).
%         \spacewidth is width of area between left and right boxes.
%
%\newlength{\rcollength}\setlength{\rcollength}{1.85in}%
\newlength{\rcollength}\setlength{\rcollength}{1.95in}%
\newlength{\spacewidth}\setlength{\spacewidth}{20pt}
%
\begin{tabular}[t]{@{}p{\textwidth-\rcollength-\spacewidth}@{}p{\spacewidth}@{}p{\rcollength}}%

% Address box
\parbox{\textwidth-\rcollength-\spacewidth}{%
D.Sc. in Computer Science\\
Address: %Rua João Bião de Cerqueira, 251, Pituba,\\
Manaus, AM, 69060-000, Brazil%\\
%Address 2: %Rua Vicente de Paula, 45, Paraíso,\\
%Guanambi, BA, 46430-000, Brazil
}

&
% Uncomment to add a vertical bar in middle of contact information
%{\vrule width 0.5pt}
\parbox[m][5\baselineskip]{\spacewidth}{} &

% Non-snail-mail contact information
\parbox{\rcollength}{%
%\textit{Fax:} +1-480-965-6899 \\
\begin{tabular}{lll}
&\textit{Mobile:} &+55-71-99268-2584 \\
&\textit{E-mail:}&\email{hugobraga@alumni.usp.br}\\
& &\email{hugo@hugobraga.io}\\
&\textit{WWW:} &\href{http://hugobraga.io}{hugobraga.io}
\end{tabular}
%\textit{WWW:} \href{http://ime.usp.br/~hbraga}{ime.usp.br/$\sim$hbraga}}
}
\end{tabular}

%%
%% In modern CV's, it seems like ``Objective'' is frowned upon. Instead,
%% incorporate it into a well-constructed cover letter. The ``More
%% information'' can go at the end of the CV, but it should not distract
%% from the section giving references available to contact.
%%
%
% \section{Objective}
%
% Placement in an academic position (i.e., faculty, postdoctoral, or
% research scientist) that allows for advanced research in distributed
% complex adaptive systems (i.e., modeling, analysis, design, and
% verification) with a particular focus on the control of engineered
% agents (e.g., for communications, control, software, electronics, and
% sustainability) and the analysis of biological phenomena (e.g.,
% self-organization, ecological rationality)
% \begin{innerlist}
% \item More information and auxiliary documents can be found at\\\url{http://www.tedpavlic.com/facjobsearch/}
% \end{innerlist}

\section{Research Interests}

\textbf{Internet of Things and Combinatorial Optimization and Graph Theory.}
%\textbf{Complex adaptive systems in control systems engineering and
%behavioral science:} distributed algorithms, computational agent-based
%modeling, hybrid dynamic systems, decentralized decision making,
%emergence and self organization, amorphous computing, autonomous
%systems, control, communications, verification, cooperation,
%optimization, game theory, resource allocation, parallel computation,
%robotics, energy systems, sustainability in the built environment,
%behavioral ecology, engineering education, bio-mimicry and
%bio-inspiration

%\section{Academic Appointments}
%
%\textbf{Postdoctoral Scholar} \hfill {July 2012 to present}
%\begin{innerlist}
%
%    \item[] \href{http://sols.asu.edu/}{School of Life Sciences},
%            \href{http://www.asu.edu/}{Arizona State University}
%    \begin{innerlist}
%        \item Supervisor: \href{http://www.public.asu.edu/~spratt1}{Professor Stephen C.~Pratt}
%        \item Decentralized decision making and behavioral bio-mimicry
%            of social insects
%        \item Affiliations (complexity science, multi-robot systems, and informational origins of life):
%            \begin{innerlist}
%                \item \href{http://csdc.asu.edu/}{Center for Social Dynamics and Complexity}\\
%                    (Directors: Michael C.~Barton, Manfred Laubichler)
%                \item \href{http://casi.asu.edu/home}{Complex Adaptive Systems @ ASU}
%                \item \href{http://faculty.engineering.asu.edu/acs/}{Autonomous Collective Systems Laboratory} (PI: Spring Berman)
%                \item \href{http://emergence.asu.edu/}{Emergence@ASU} (PI: Paul Davies, co-PI: Sara Walker)
%            \end{innerlist}
%    \end{innerlist}
%
%\end{innerlist}
%
%\halfblankline
%
%\textbf{Postdoctoral Researcher} \hfill {September 2010 to June 2012}
%\begin{innerlist}
%
%    \item[] \href{http://www.cse.ohio-state.edu/}{Department of Computer Science and Engineering},
%            \href{http://www.osu.edu/}{The Ohio State University}
%    \begin{innerlist}
%        \item \href{http://www.nfs.gov/}{National Science Foundation} Cyber-Physical Systems (ENG, \href{http://www.nsf.gov/div/index.jsp?div=eccs}{ECCS})
%        \begin{innerlist}
%            \item[$-$] ``Autonomous Driving in Mixed-Traffic Urban Environments''
%                (grant~\href{http://www.nsf.gov/awardsearch/showAward.do?AwardNumber=0931669}{\#0931669})
%            \item[$-$] Supervisor (co-PI):
%                \href{http://www.cse.ohio-state.edu/~paolo/}%
%                     {Professor Paolo A.~G.~Sivilotti}
%            \item[$-$] PI:
%                \href{http://www.ece.ohio-state.edu/~umit/}%
%                     {Professor \"{U}mit \"{O}zg\"{u}ner}
%        \end{innerlist}
%    \end{innerlist}
%
%\end{innerlist}

\section{Education}

\href{http://www.usp.br/}{\textbf{University of São Paulo}},
São Paulo, SP, Brazil
\begin{outerlist}

\item[] D.Sc.,
        \href{http://www.ime.usp.br/dcc/}
             {Computer Science},
             December 2018 \cite{Braga2018}
        \begin{innerlist}
        \item Thesis Topic: \emph{Exact algorithms for spanner problems in graphs (in Portuguese)}
        %\item Thesis Proposal: \emph{Cooperative Task Processing}
        %\item Candidacy: \emph{Research Problems in Distributed Control for Energy Systems}
        \item Superviser:
              \href{http://www.ime.usp.br/~yw/}
                   {Professor Yoshiko Wakabayshi}
        \item Area of Study: Combinatorial Optimization and Graph Theory
        \end{innerlist}

\end{outerlist}

\halfblankline

\href{http://www.usp.br/}{\textbf{Federal University of Bahia}},
Salvador, BA, Brazil
\begin{outerlist}

\item[] M.Sc.,
        \href{http://wiki.dcc.ufba.br/Mecatronica/}
             {Mechatronics}, October 2012 \cite{Braga2012}
        \begin{innerlist}
        \item Thesis Topic: \emph{Algorithms for the Directed k-Spanner with Minimum Degree Steiner Tree Problem}
        \item Superviser:
              \href{http://wiki.dcc.ufba.br/DCC/ProfFlavioAssis}
                   {Professor Fl\'{a}vio Assis}
        \item Area of Study: Mechatronics, Algorithms and Graph Theory
        \end{innerlist}

\item[] PgCert,
        \href{http://www.lasid.ufba.br/easd}
             {Distributed Systems}, December 2009
        \begin{innerlist}
        \item Final Project Topic: \emph{Medium Accesss Control Protocols for Wireless Sensor Networks} (in Portuguese)
        \item Superviser:
              \href{http://wiki.dcc.ufba.br/DCC/ProfFlavioAssis}
                   {Professor Fl\'{a}vio Assis}
        \item Area of Study: Medium Access Control, Algorithms
        \end{innerlist}

\item[] B.Sc.,
        \href{http://wiki.dcc.ufba.br/DCC/}
             {Computer Science}, December 2008
        \begin{innerlist}
	\item Final Project Topic: \emph{Design and Implementation of a Monitoring Mechanism for a QoS Distributed Provider } (in Portuguese)
%        \item \emph{Magna cum Laude}, With Honors in Engineering
%        \item Electrical specialization (emphasis on electromagnetics and digital computers)
%        \item Minor in \href{http://www.cse.ohio-state.edu/}
%                            {Computer and Information Systems}
%              (programming and algorithms)
	\item Superviser:
              \href{http://wiki.dcc.ufba.br/DCC/ProfSergioGorender}
                   {Professor S\'{e}rgio Gorender}
        \item Area of Study: Quality of Service, Real-Time Systems
        \end{innerlist}

\end{outerlist}

\nocite{*}

 \section{Publications}
%
% % Add a little space to nudge next ``Ref'd Journal Publications'' marginpar
% % down to make room for tall ``Submitted Journal Publications''
% % marginpar. If there are enough submitted journal publications, this
% % space will not be needed (and should be removed).
 % \vspace{0.1in}

 %% \newrefcontext[labelprefix=P]
 %% \printbibliography[heading=none,keyword=publication,resetnumbers=true]
%\begin{refsection}
 \newrefcontext[labelprefix=P]
 \printbibliography[
%   title = {Publications},
 heading=none,
 keyword=publication,
 resetnumbers=true
]
%\end{refsection}
 
 %\addcontentsline{toc}{section}{Journal Publications}
%\vspace{0.1in}

 %% \assignrefcontextkeyws[labelprefix=O]{O}
%% \assignrefcontextkeyws[labelprefix=G]{G}

 
\section{Other Publications}

%% \begin{refcontext}[labelprefix=O]
%\begin{refsection}
\newrefcontext[labelprefix=O]
 %% \newrefcontext[labelprefix=O]
 %% \printbibliography[heading=none,keyword=others,resetnumbers=true]
 
\printbibliography[
%  title = {Other Publications},
 heading=none,
 keyword=others,
 resetnumbers=true
 ]
%\end{refsection}

\section{Grants}

%% \begin{refcontext}[labelprefix=O]
%\begin{refsection}
\newrefcontext[labelprefix=G]
 %% \newrefcontext[labelprefix=G]
 %% \printbibliography[heading=none,keyword=grant,resetnumbers=true]

\printbibliography[
%  title = {Grants},
 heading=none,
 keyword=grant,
 sorting=adnyvt,
 resetnumbers=true
 ]
%\end{refsection}


%% \end{refcontext}

%~ \section{Academic Service}
%~ 
%~ \begin{bibsection}
%~ 
    %~ \item \textbf{\emph{Committee for the Development of Biomimicry and
        %~ Bio\-/inspired Research and Education Initiatives at ASU}}\\
        %~ Chairman, Arizona State University. 2013.
%~ 
    %~ \item \textbf{\emph{Interdisciplinary Complexity Science Stduent
        %~ Organization}}\\
        %~ Founding faculty co\-/adviser, Arizona State University.
        %~ Interdisciplinary graduate and undergraduate student group
        %~ focused on discussion of research topics in complexity science.
        %~ 2013.
%~ 
%~ \end{bibsection}
%~ 
%~ \section{Student Advising}
%~ 
%~ \begin{bibsection}
%~ 
    %~ \item \textbf{Hana Putnam} and \textbf{Alex Nachman}\\
        %~ Undergraduate students in Biology, Arizona State University.
        %~ Laboratory support of research on decentralized nutrient
        %~ regulation in \emph{Temnothorax rugatulus} ants.
        %~ Primary adviser: Stephen C.~Pratt.
        %~ 2013.
%~ 
    %~ \item \textbf{Taylor Vance} and \textbf{P.~Logan Rogers}
        %~ and \textbf{Betsy Siegworth}\\
        %~ Undergraduate students in Biology, Arizona State University.
        %~ Laboratory support of research on quorum detection by encounter
        %~ rate in \emph{Temnothorax rugatulus} ants.
        %~ Primary adviser: Stephen C.~Pratt.
        %~ 2013.
%~ 
    %~ \item \textbf{Sean T.~Wilson}\\
        %~ Graduate student in Mechanical Engineering, Arizona State University.
        %~ Dynamical modeling and analysis of the collective carrying
        %~ behaviors of \emph{Aphaenogaster cockerelli} ants.
        %~ Primary adviser: Spring M.~Berman.
        %~ 2012--2013.
%~ 
    %~ \item \textbf{Ganesh P.~Kumar}\\
        %~ Graduate student in Computer Science, Arizona State University.
        %~ Bio\-/mimetic design of collective carrying algorithms for
        %~ robotics, inspired by the ant \emph{Aphaenogaster cockerelli}.
        %~ Primary adviser: Spring M.~Berman.
        %~ 2012--2013.
%~ 
    %~ \item \textbf{Christal Johnson}\\
        %~ Undergraduate student in Biology, Arizona State University.
        %~ Modeling and analysis of quorum detection during emigration
        %~ behavior in \emph{Temnothorax rugatulus} ants. Honors thesis.
        %~ Primary adviser: Stephen C.~Pratt.
        %~ 2012.
%~ 
    %~ \item \textbf{Cory Henderson}, \textbf{James O'Donnell},
        %~ \textbf{Ian Neack}, and \textbf{Patrick Whewell}\\
        %~ Undergraduate students in Electrical and Computer Engineering.
        %~ Group design project on retrofittable
        %~ vehicle\-/to\-/vehicle communications system for
        %~ adaptive\-/cruise\-/control in mixed\-/traffic environments.
        %~ Primary adviser: Keith Redmill.
        %~ 2012.
%~ 
    %~ \item \textbf{Manas Agrawal}
        %~ Graduate student in Computer Science and Engineering.
        %~ Software verification and model checking applied to railroad
        %~ safety problems.
        %~ Primary adviser: Bruce W.~Weide.
        %~ 2012.
%~ 
    %~ \item \textbf{Sai Prathyusha Peddi}
        %~ Graduate student in Computer Science and Engineering.
        %~ Software verification applied to adaptive cruise
        %~ control and instrumented intersection signal timing.
        %~ Primary adviser: Bruce W.~Weide.
        %~ 2011--2012.
%~ 
    %~ \item \textbf{Jaeyong Park}.
        %~ Graduate student in Electrical and Computer Engineering.
        %~ Provably correct on\-/line control synthesis for
        %~ autonomous vehicles with hybrid dynamics.
        %~ Primary adviser: \"{U}mit \"{O}zg\"{u}ner.
        %~ 2011--2012.
%~ 
%~ \end{bibsection}

\section{Teaching Experience}

\href{http://www.ufba.br/}{\textbf{Federal University of Bahia}},
Salvador, BA, Brazil
\begin{outerlist}

\item[] \textit{Assistant Lecturer} for MATA86: Computer Networks
    \hfill \textbf{August~2011 to December~2011}
    \begin{innerlist}
        \item Graduate-level and Undergraduate-level course in algorithms for 
Wireless Sensor Networks
    \end{innerlist}

\end{outerlist}

\halfblankline

\href{http://ifba.edu.br/}{\textbf{Federal Institute of Education, Science and Technology of Bahia (IFBa)}},
Salvador, BA, Brazil
\begin{outerlist}

\item[] \textit{Guest Lecturer} for Advanced Topics in Computer Science course
    \hfill \textbf{August~2011}
    \begin{innerlist}
        \item Undergraduate-level course
        \item Lecture: ``Topology control algorithms for Wireless Sensor Networks''
    \end{innerlist}
\end{outerlist}

%~ \section{Professional Service}
%~ 
%~ \textbf{Committee Service}
%~ \begin{innerlist}
    %~ \item Officer, IEEE Special Technical Community for Human Computation
%~ \end{innerlist}
%~ 
%~ \halfblankline
%~ 
%~ \textbf{Referee Service}
%~ \begin{innerlist}
    %~ \item \emph{49\textsuperscript{th} Annual Conference on Decision and Control}
    %~ \item \emph{International Journal of Control}
    %~ \item \emph{ASME Journal of Dynamic Systems, Measurement, and Control}
    %~ \item \emph{IEEE Transactions on Signal Processing}
    %~ \item \emph{IEEE Transactions on Control Systems Technology}
    %~ \item \emph{IEEE Transactions on Cybernetics}
    %~ \item \emph{IEEE Transactions on Intelligent Transportation Systems}
    %~ \item \emph{The International Journal of Robotics Research}
    %~ \item \emph{Engineering Applications of Artificial Intelligence}
    %~ \item \emph{Bioinspiration \& Biomimetics}
    %~ \item \emph{Swarm and Evolutionary Computation}
    %~ \item \emph{Behavioral Ecology}
    %~ \item \emph{Ecological Research}
    %~ \item \emph{Journal of Theoretical Biology}
%~ \end{innerlist}
%~ 
%~ \halfblankline
%~ 
%~ \textbf{Editorial Service}
%~ \begin{innerlist}
    %~ \item \emph{Human Computation}, editorial board (2014--)
%~ \end{innerlist}
%~ 
%~ \halfblankline
%~ 
%~ \textbf{Conference Service}
%~ \begin{bibsection}[\enskip\textbullet,leftmargin=*]
    %~ \item Co\-/organizer (with Yun Kang) for technical session:
        %~ ``Complex Systems of Social Insects in Research and Education'',
        %~ 2013 International Symposium on Biomathematics and Ecology
        %~ Education and Research~(BEER~2013), Arlington, VA, October
        %~ 11--13, 2013.
%~ 
    %~ \item Organizer for mini\-/symposium: ``MS19: Optimization and
        %~ Rationality in Eusocial Insects'', 2013 Society for Mathematical
        %~ Biology Annual Meeting and Conference~(SMB~2013), Tempe, AZ,
        %~ June 10--13, 2013.
%~ 
    %~ \item Organizer/Associate Editor for invited session: ``Correctness
        %~ by Verification and Design'', 14\textsuperscript{th} IEEE
        %~ Conference on Intelligent Transportation Systems~(ITSC~2011),
        %~ Washington, DC, October 5--7, 2011.
%~ \end{bibsection}

\section{Professional Experience}
%editing this section
\href{https://www.tjam.jus.br/}{\textbf{Amazonas Court of Justice (TJAM)}},
Manaus, Brazil
\begin{outerlist}

\item[] \textit{Analyst~-~Software Engineer}%
        \hfill \textbf{August 2020 up to now}
\begin{innerlist}
\item Ranked 2º in the public service entrance exam for Analyst position~-~Speciality: Software Engineer.
\item Member of Secretariat of Information and Communication Tecnology (SETIC).
\end{innerlist}

\item[] \textit{Project Manager}%
\hfill \textbf{December 2022 up to May 2024}
\begin{innerlist}
\item Project Contract Management.
\item Streamlining Process Flow.
\item Overseeing Software Factory Projects.
\item Managing Remote Teams.
\item Leading Internal Project Development.
\end{innerlist}

\end{outerlist}

\halfblankline

\href{http://www.usp.br}{\textbf{University of S\~{a}o Paulo}},
S\~{a}o Paulo, SP, Brazil
\begin{innerlist}
  \item[] 
    Institute of Mathematics and Statistics\\
    Computer Science Department\\
    Combinatorics and Combinatorial Optimization Research Group
\end{innerlist}

\begin{outerlist}

\item[] \textit{Graduate Researcher}~\cite{Braga2017,Braga2018}%
  \hfill \textbf{August 2013 to December 2018}
  \begin{innerlist}
    
  \item Funding:~\cite{Grant2014}
  \item Supervisor: \href{http://www.ime.usp.br/~yw}%
    {Professor Yoshiko Wakabayashi}
  \item Research: Combinatorial Optimization and Graph Theory.
  \item Proposal of the first two ILP formulations for the \emph{Minimal Weight Tree $t$-Spanner Problem}.
  \item Proposal of an ILP formulation for the \emph{Minimal Weight $t$-Spanner Problem} (MWSP).
  \item Presentment of some computational results on the implementation of a branch and price algorithm for MWSP based on an ILP formulation proposed by Sigurd and Zachariasen (2004).  

  \end{innerlist}

\end{outerlist}

\halfblankline

\href{http://www.ufba.br}{\textbf{Federal University of Bahia}},
Salvador, BA, Brazil

\begin{innerlist}
  \item[]
    Computer Science Department\\
    Distributed Systems Laboratory (LaSiD)
\end{innerlist}

\begin{outerlist}

    \item[] \textit{Graduate Researcher}~\cite{Braga2011,Braga2012}%
            \hfill \textbf{March 2010 to April 2013}
            \begin{innerlist}
                \item Funding:~\cite{Grant2012,Grant2010}
                \item Supervisor:
                  \href{http://wiki.dcc.ufba.br/DCC/ProfFlavioAssis}
                       {Professor Fl\'{a}vio Assis}

                \item Description of a localized topology control algorithm 
                  which is very efficient in terms of interference 
                  while minimizing energy efficiency.
                  
                \item Development of a new interference metric for Wireless Sensor Networks and evaluation of 
                some topology control algorithms based on this metric.

                \item Proposal of a new problem in graph theory literature to 
address the spanner property and degree minization called 
                   \emph{Directed k-Spanner with Minimum 
                    Degree Steiner Tree Problem} (DSMDStP).
                 \item Development of an approximation algorithm and a heuristic 
				for DSMDStP problem.
            \end{innerlist}
            
		\item[] \textit{Undergraduate Researcher}~\cite{Braga2008}%
            \hfill \textbf{August 2007 to July 2008}
            \begin{innerlist}
                \item Funding:~\cite{Grant2007b}
                \item Supervisor:
                  \href{http://wiki.dcc.ufba.br/DCC/ProfSergioGorender}
                   {Professor S\'{e}rgio Gorender}
                \item Designed and implemented a monitoring mechanism for a QoS
 Distributed Provider using C language and SNMP protocol, and the mechanism was
 applied in Xenomai Real-Time Operating System and the Expedited Services of 
Cisco routers.
            \end{innerlist}

\end{outerlist}

\halfblankline

\href{http://www.prodeb.ba.gov.br}{\textbf{Data Processing Company of the State of Bahia (Prodeb)}},
Salvador, Brazil
\begin{outerlist}

\item[] \textit{Software Engineer}%
        \hfill \textbf{February 2009 to February 2010}
\begin{innerlist}
\item Development web services using PHP language.
\item Co-development a calendar system using PHP language along with Postgres DB 
and script languages.
\item Wrote (English) tutorials for the open source community as well as 
documentation for the developed systems.
\end{innerlist}
\end{outerlist}

\halfblankline

\href{http://www.geotecnia.ufba.br/}{\textbf{Federal University of Bahia}},
Salvador, BA, Brazil

\begin{innerlist}
  \item[]
    Polytechnic school\\
    Geotechnical Laboratory
\end{innerlist}

\begin{outerlist}

\item[] \textit{Programmer}%
        \hfill \textbf{January 2007 to May 2007}
\begin{innerlist}
\item Funding:~\cite{Grant2007a}
\item Maintenance and development of a system in DELPHI for sample management and calculation of soil tests called LabGeo.
\item Support the development of a web system for a weather station using PHP
 and MySQL languages.
\end{innerlist}

\end{outerlist}

\section{Professional Memberships}

%Institute for Electrical and Electronics Engineers~(IEEE), Member,
%2002--present
%
\begin{innerlist}
\item \href{http://www.redebrasileira.org}{Brazilian Network of Smart and Human Cities - RBCIH} (05/2020--today)
\item \href{https://abinc.org.br/conexao-iot}{The Internet of Things Brazilian Association - IoT Connection} (04/2020--today)    
\item \href{http://www.ime.usp.br/~yoshi/index_combinatorics.html}{Combinatorics and Combinatorial Optimization Research Group} (08/2013--12/2018)
\item \href{http://www.lasid.ufba.br/}{Distributed Systems Laboratory (LaSiD)} (08/2007--12/2008; 03/2010--06/2013)
\item \href{http://www.sbc.org.br/en/}{Brazilian Computer Society (SBC)} (2007--2009)
\end{innerlist}

%\halfblankline
%
%Animal Behavior Society, Member, 2011--present
%
%\halfblankline
%
%International Union for the Study of Social Insects, Member, 2012--present
%\begin{innerlist}
%\item North American Section (2012--present)
%\end{innerlist}
%
%\halfblankline
%
%Entomological Society of America, Member, 2014--present
%\begin{innerlist}
%\item Southwestern and Pacific Branch (2014--present)
%\item Systematics, Evolution, and Biodiversity Section (2014--present)
%\end{innerlist}
%
%\halfblankline
%
%Society for Mathematical Biology, Member, 2012--present

\section{Other Meeting Attendance}

\begin{innerlist}
\item \href{https://sebrae.com.br/startupday}{Sebrae - Startup Day, May 22, 2021}
  
\item \href{https://telecomwebinar.com/iot-brasil-summit-2020/}{IoT 2020 Brasil Summit, June 03, 2020}
  
    \item \href{http://csbc2017.mackenzie.br/eventos/2-etc}{II Meeting of Theory of Computing (XXXVII CSBC), July 03--04, 2017}
  
    \item \href{https://www.ic.unicamp.br/~ra134042/wopoca2017/}{I Paulista Workshop on Optimization, Combinatorics and Algorithms, June 16--18, 2017}
  
    \item \href{http://org.nicta.com.au/study-with-us/nicta-optimisation-summer-school-2015/}{NICTA 3rd International Optimisation Summer School, January 11--16, 2015}

	\item \href{http://www.ime.usp.br/~afreire/WBA2014.html}{Workshop in Bioinformatics and Algorithms, March 25--26, 2014}

    \item \href{http://www.lasid.ufba.br/disc2012/view/index.php}{26\textsuperscript{th} International Symposium on Distributed Computing, October 16--18, 2012}

    \item \href{http://softwarelivre.org/fisl}{10\textsuperscript{th} International Free Software Forum, June 24--27, 2009}
    
    \item \href{http://www.erbase2008.ufba.br}{VIII Computer Regional School of Bahia, Alagoas and Sergipe (ERBASE), April 14--18, 2008}

    \item IX Research and Graduate Seminar -- XXVII Student Research Seminar/PRPPG -- UFBA, Salvador, Brazil, November 12--14, 2008

    \item \href{https://gts.nic.br/reunioes/gts-11}{GTER 20 and GTS 11 joint meetings, May 31--June 1, 2008}

    \item \href{http://listas.dcc.ufba.br/pipermail/estudantes-comp/2007-November/003076.html}{XI Internal Workshop of LaSiD -- Wola, November 27, 2007}

    \item \href{http://wiki.enec.org.br/ENECOMP2004}{XXII National Meeting of Computer Science Students, 2004}

\end{innerlist}

%~ \section{Service}
%~ 
%~ Intel International Science and Engineering Fair (ISEF) 2013
%~ \begin{innerlist}
    %~ \item Grand Award Judge for Animal Sciences
%~ \end{innerlist}
%~ 
%~ \halfblankline
%~ 
%~ \href{http://opendoor.asu.edu}{Night of the Open Door},
%~ \href{http://www.asu.edu}{Arizona State University},
%~ 2013
%~ %
%~ \begin{innerlist}
    %~ \item Staffed the ``Ants of Arizona'' exhibit
    %~ \item Answered questions about ants and research related to them
%~ \end{innerlist}
%~ 
%~ \halfblankline
%~ 
%~ Recent contributor to several open\-/source software projects, including:
%~ \begin{innerlist}
    %~ \item \href{http://vim-latex.sourceforge.net/}{Vim\-/LaTeX} suite
    %~ \item \href{http://vimperator.org}{Vimperator} and
        %~ \href{http://dactyl.sourceforge.net/pentadactyl/index}{Pentadactyl}
        %~ Firefox extensions
    %~ \item \href{http://git-scm.com}{Git} distributed version control
        %~ system
    %~ \item \href{http://www.selenic.com/mercurial/}{Mercurial} distributed version control
        %~ system
    %~ \item Personal projects archived at
        %~ \url{http://hg.tedpavlic.com/}
%~ \end{innerlist}
%~ 
%~ \halfblankline
%~ 
%~ Frequent contributor to \href{http://www.wikipedia.org/}{Wikipedia}
%~ %
%~ \begin{innerlist}
    %~ \item Significant contributions to articles on control theory,
        %~ electronics, and signals and systems.
%~ \end{innerlist}
%~ 
%~ \halfblankline
%~ 
%~ Contributor to \href{http://www.quora.com/}{Quora}
%~ %
%~ \begin{innerlist}
    %~ \item Contributions to articles on thermodynamics, chaos theory,
        %~ electronics, and evolutionary biology.
%~ \end{innerlist}
%~ 
%~ \halfblankline
%~ 
%~ \href{http://www.osufirst.org/}{OSU FIRST Robotics Team},
%~ \href{http://www.osu.edu}{The Ohio State University}, 2000--2004
%~ \begin{innerlist}
%~ \item Introduced middle school and high school students to science and
        %~ technology by participating with them in national robotics
        %~ competitions.
%~ \item Led 2002 team to regional silver medal
        %~ \href{http://www.firstwiki.org/Engineering_Inspiration_Award}
             %~ {\emph{Engineering Inspiration Award}}.
%~ \item \emph{Lead Team Mentor}, 2002--2004
%~ \item \emph{Component Design Team Lead Mentor}, 2001--2002
%~ \end{innerlist}
%~ 
%~ \halfblankline
%~ 
%~ Ohio Science Olympiad state competition, Robot Ramble Event, 2003
%~ %
%~ \begin{innerlist}
    %~ \item Supervised setup and judging of event for middle-school and
        %~ high-school students
%~ \end{innerlist}
%~ 
%~ \halfblankline
%~ 
%~ Director of Computers,
%~ \href{http://ec.osu.edu/}{Engineers' Council},
%~ \href{http://www.osu.edu/}{The Ohio State University}, 2002
%~ 
%~ \halfblankline
%~ 
%~ \href{http://www.linuxvirtualserver.org/}
     %~ {Linux Virtual Server Project}, 1999--2000
%~ \begin{innerlist}
%~ \item Early member of the team that formed the open\-/source project that
        %~ is now an important load balancing solution for the Linux
        %~ software platform.
%~ \end{innerlist}
%~ 
%~ \halfblankline
%~ 
%~ \href{http://www.gcfn.org/}
     %~ {Greater Columbus Free\-/Net}, 1995--1997
%~ \begin{innerlist}
%~ \item Provided technical support services.
%~ \end{innerlist}
%~ 
%~ \halfblankline
%~ 
%~ CompuTeen Bulletin Board System, 1993--1995
%~ \begin{innerlist}
%~ \item Administrated dial\-/up bulletin board system.
%~ \item Founded and administrated TeenLiNK, an international electronic
        %~ mail network that spread through the United States, Canada, and
        %~ Australia and delivered mail over a series of electronic dial\-/up
        %~ drop offs.
%~ \end{innerlist}
%~ 
%~ \section{Application Areas}
%~ 
%~ Autonomous and Unmanned Vehicles, Flexible Manufacturing Systems,
%~ Distributed Power Generation, Intelligent Lighting, Power Demand
%~ Response, Microgrids, Smart Grids

%\section{Hardware and Software Skills}
\section{Skills}

Languages
%
\begin{innerlist}
    \item Portuguese (Native)
    \item English (advanced speaking and writing skills~-~\href{https://drive.google.com/file/d/1gUh4rYdd2md1kp8KcbMT-JoJne5NWcQx/view?usp=sharing}{TOEFL iBT: 101, taken on December 2012})
    \item Basic French
\end{innerlist}

\halfblankline

Program/Script Languages
%
\begin{innerlist}
    \item C/C++, Java, Python, Shell Script, PHP
\end{innerlist}

\halfblankline

Scientific Softwares
%
\begin{innerlist}
    \item ILP Solver: CPLEX
    \item Simulator/Emulator: The ONE, Omnet++, Castalia, SMPL
    \item Real-Time: Xenomai, Diffserv (Cisco Routers)
    \item Others: SNMP, Lemon (c++)
\end{innerlist}

%~ \section{Expertise}
%~ 
%~ Mathematics:
%~ %
%~ \begin{innerlist}
    %~ \item Applied Mathematics, Real and Complex Analysis, Measure
        %~ Theory, Differential Geometry, Game Theory, Graph Theory,
        %~ Combinatorics
%~ \end{innerlist}
%~ 
%~ \halfblankline
%~ 
%~ Control Theory and Engineering:
%~ %
%~ \begin{innerlist}
    %~ \item Linear and Nonlinear Systems Theory, Feedback, Variable
        %~ Structure Systems and Sliding Modes, Distributed and Intelligent
        %~ Control, Dynamic Optimization, Biomimicry, Bioinspiration,
        %~ Hybrid and CyberPhysical Systems
%~ \end{innerlist}
%~ 
%~ \halfblankline
%~ 
%~ Communications and Signal Processing:
%~ %
%~ \begin{innerlist}
    %~ \item Probability, Random Variables, Stochastic Processes,
        %~ Information Theory, Estimation, Networks
%~ \end{innerlist}
%~ 
%~ \halfblankline
%~ 
%~ Computer Science and Engineering:
%~ %
%~ \begin{innerlist}
    %~ \item Model Checking (automated, distributed, hybrid,
        %~ probabilistic), Hybrid Automata, Software Verification,
        %~ Component\-/Based Reusable Software
%~ \end{innerlist}
%~ 
%~ \halfblankline
%~ 
%~ Natural and Social Sciences (Biology, Neuroscience, Psychology, Anthropology):
%~ %
%~ \begin{innerlist}
    %~ \item Behavioral Ecology, Foraging Theory, Altruism, Impulsiveness,
        %~ Evolution
%~ \end{innerlist}
%~ 
%~ \section{Awards}
%~ 
%~ \href{http://www.nsf.gov/}{National Science Foundation}
%~ \begin{innerlist}
%~ \item \href{http://www.nsfgk12.org/}{GK\-/12 Graduate Fellowship}, 2006--2007
%~ \item \href{http://www.nsf.gov/grfp}
           %~ {Graduate Research Fellowship} Honorable Mention, 2005
%~ \end{innerlist}
%~ 
%~ \halfblankline
%~ 
%~ \href{http://www.osu.edu}{The Ohio State University}
%~ \begin{innerlist}
%~ \item \href{http://www.gradsch.osu.edu/graduate-school-fellowships-for-first-year-graduate-students.html}
           %~ {Dean's Distinguished University (DDU) Graduate Fellowship}, 2004--2010
%~ \item Electrical and Computer Engineering Bradshaw Scholarship,
        %~ 2002--2004
%~ \item Electrical and Computer Engineering Shafstall Scholarship,
        %~ 2001--2003
%~ \item University Scholarship, 1999--2003
%~ \end{innerlist}
%~ 
%~ \section{Security Clearance}
%~ 
%~ Department of Defense Top Secret SCI with polygraph (expired: 2002)

% \section{Citizenship}
%
% USA

\section{References Available to Contact}

\href
{http://wiki.dcc.ufba.br/DCC/ProfFlavioAssis}
{\textbf{Prof. (Dr.-Ing) Fl\'{a}vio Assis}}
(e-mail:~\href{mailto:fassis@ufba.br}{fassis@ufba.br})
%
\begin{innerlist}
    \item Associate Professor,
        \href{http://wiki.dcc.ufba.br/DCC/}{Departament of Computer Science},
        \href{http://www.ufba.br/}{Federal University of Bahia}

    %\item[$\diamond$] School of Life Sciences, PO Box 874501, Tempe, AZ
    %    85287-4501

    \item[$\star$] \emph{Dr.~Assis was my MSc. thesis supervisor.}
\end{innerlist}

\halfblankline

\href
{http://www.ime.usp.br/~yw/}
{\textbf{Prof.~(Dr. rer. nat.) Yoshiko Wakabayashi}}
(e-mail:~\href{mailto:yw@ime.usp.br}{yw@ime.usp.br})
%
\begin{innerlist}
    \item Full Professor,
        \href{http://ime.usp.br/dcc}{Department of Computer Science},
        \href{http://www.usp.br}{University of São Paulo}

    %\item[$\diamond$] School for Engineering of Matter, Transport, and
    %    Energy, PO Box 876106, Tempe, AZ
    %    85287-6106

    \item[$\star$] \emph{Dr.~Wakabayashi was my PhD. thesis supervisor.}
\end{innerlist}


\halfblankline

\href
{http://www.macedo.ufba.br}
{\textbf{Prof.~(PhD.) Raimundo J.~A.~Macêdo}}
(e-mail:~\href{mailto:macedo@ufba.br}{macedo@ufba.br})
%
\begin{innerlist}
    \item Full Professor,
        \href{http://wiki.dcc.ufba.br/DCC/}{Departament of Computer Science},
        \href{http://www.ufba.br/}{Federal University of Bahia}

    %\item[$\diamond$] 205 Dreese Laboratories, 2015 Neil Ave., Columbus,
    %    OH  43210

    \item[$\star$] \emph{Dr.~Macêdo is the head of LaSiD.}
\end{innerlist}

\halfblankline

\href
{http://wiki.dcc.ufba.br/DCC/ProfSergioGorender}
{\textbf{Prof.~(Dr.) Sérgio Gorender}}
(e-mail:~\href{mailto:gorender@ufba.br}{gorender@ufba.br})
\begin{innerlist}
    \item Associate Professor,
        \href{http://wiki.dcc.ufba.br/DCC/}{Departament of Computer Science},
        \href{http://www.ufba.br/}{Federal University of Bahia}

    %\item[$\diamond$] 395 Dreese Laboratories, 2015 Neil Ave., Columbus,
    %    OH  43210

    \item[$\star$] \emph{Dr.~Gorender was my B.Sc. supervisor. I did my Scientific Initiation under his supervisor.}
\end{innerlist}

% The ``More Info'' section may not be necessary; make sure it's short
% so it doesn't prevent people from seeing references available to
% contact.
\section{More Information}

More information and auxiliary documents can be found at\\%
\url{http://hugobraga.io}.

\end{document}

%%%%%%%%%%%%%%%%%%%%%%%%%% End CV Document %%%%%%%%%%%%%%%%%%%%%%%%%%%%%

%----------------------------------------------------------------------%
% The following is copyright and licensing information for
% redistribution of this LaTeX source code; it also includes a liability
% statement. If this source code is not being redistributed to others,
% it may be omitted. It has no effect on the function of the above code.
%----------------------------------------------------------------------%
% Copyright (c) 2007, 2008, 2009, 2010, 2011 by Theodore P. Pavlic
%
% Unless otherwise expressly stated, this work is licensed under the
% Creative Commons Attribution-Noncommercial 3.0 United States License. To
% view a copy of this license, visit
% http://creativecommons.org/licenses/by-nc/3.0/us/ or send a letter to
% Creative Commons, 171 Second Street, Suite 300, San Francisco,
% California, 94105, USA.
%
% THE SOFTWARE IS PROVIDED "AS IS", WITHOUT WARRANTY OF ANY KIND, EXPRESS
% OR IMPLIED, INCLUDING BUT NOT LIMITED TO THE WARRANTIES OF
% MERCHANTABILITY, FITNESS FOR A PARTICULAR PURPOSE AND NONINFRINGEMENT.
% IN NO EVENT SHALL THE AUTHORS OR COPYRIGHT HOLDERS BE LIABLE FOR ANY
% CLAIM, DAMAGES OR OTHER LIABILITY, WHETHER IN AN ACTION OF CONTRACT,
% TORT OR OTHERWISE, ARISING FROM, OUT OF OR IN CONNECTION WITH THE
% SOFTWARE OR THE USE OR OTHER DEALINGS IN THE SOFTWARE.
%----------------------------------------------------------------------%
